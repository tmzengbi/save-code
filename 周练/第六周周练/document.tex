\documentclass[UTF8]{ctexart}
\usepackage{graphicx}
\usepackage{ulem}
\title{第六周周练题解}
\author{ZeNgBi}
\date{}
\begin{document}
	\maketitle
	\subsection*{P1000}
	第零周周练第一题,题解请回顾第零周周练题解
	\subsection*{P1001}
	贪心策略是先拿所有是11的人(假设有n个),令01为a,10为b,拿$min(a,b)$数量的最大影响力的01和10,最后将剩下的人根据影响力排序从大到小拿n个为最优解。
	\subsection*{P1002}
	\subsection*{P1003}
	贪心策略是按$l_i*r_i$从小到大排序,以下是证明。
	
	假设第$N_n$排在$N_{n+1}$前面时,有最优解,此时他们左右手数字分别为$a_1 \:b_1\: a_2 \:b_2\: $。此时令前面所有人左手数字乘积为$k=\prod_{1}^{n-1} l_i$
	
	那么$ans_1=min(\frac{k}{b_1},\frac{ka_1}{b_2})$同理如果他们反过来站那么$ans_2=min(\frac{k}{b_2},\frac{ka_2}{b_1})$ 如果$ans_1 < ans_2$ 有$\frac{ka_1}{b_2}<\frac{ka_2}{b_1}$即$a_1*b_1<a_2*b_2$
	
	数字会很大注意写高精
	\subsection*{P1004}
	每行至少要区一个黑块,那么这一行没有选过且只有一个黑块时,只能选它,如果某一行没有选过且这一行上面没有黑块了那么无法完成。如果最后没有选过的行都至少有两个或以上个黑块,必然可以选出一个结果。
	\subsection*{P1005}
	记录砍能造成的最大伤害为k,并将伤害比k大的丢记录下来,和为sum。
	
	然后一直砍怪物,直到可以一套丢连击可以秒掉它。
	
	\subsection*{P1006}
	答案是Yes,详细证明见ppt讲解
	
	\subsection*{P1007}
	首先可以可以证明单调序列答案一定小于摆动序列
	
	摆动序列中间的贡献大于两边,所以中间数大的的要尽可能大,小的要尽可能小,摆动方向为$\uparrow \downarrow \uparrow$和方向为$\downarrow \uparrow \downarrow$是两种情况,要分别讨论。
\end{document}