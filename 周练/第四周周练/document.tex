\documentclass[UTF8]{ctexart}
\usepackage{graphicx}
\title{第四周周练题解}
\author{ZeNgBi}
\date{}
\begin{document}
	\maketitle
	\subsection*{P1000 蒟蒻单词接龙}
	预处理两个单词之间能否接龙,然后暴力搜索
	\subsection*{P1001 蒟蒻树}
	中序遍历是以左->根->右的方式遍历一棵树
	
	于是我们可以令i为根$1\le i\le n$那么i左边的节点为i的左子树,右边的节点为i的右子树
	
	为了找到最大的加分,必须从1-n枚举i,即dfs(1,n),然后先从左子树找根,即从1-(i-1)枚举搜索找根,在从右子树找根
	
	对于每次搜索计算加分,最大分数为score=max(dfs(1,i-1)*dfs(i+1,n)+root[i])。但是我们每次计算都是用的子树的最大的分数所以可以\textbf{记忆化搜索},对于中序遍历为i-j的子树score[i][j]=max(dfs(i,k-1)*dfs(k+1,j)+root[k])($i\le k\le j$)
	
	边界条件为 i>j时表示为空子树 return 1;i==j时表示为叶子树 return root[i];
	 
	前序遍历输出就把搜索到最大值时候的根纪录下来输出。
	\subsection*{P1002 活蹦乱跳的木薯马}
	因为要求最少走几步,考虑广搜。
	
	先将起点放如队列,向八个方向尝试跳跃,如果在界内且该方格还没有访问过,更新节点信息,从新放入队列。
	\subsection*{P1003 涂色游戏}
	要求一个封闭空间,对于空地我们不知道它在圈内还是圈外,而且圈外的空间不一定连通,如何搜索?
	
	可以在把n*n的方阵扩大为(n+2)*(n+2),相当于把方阵扩大一圈,那么圈外的空间就可以通过外面增加的一圈连通,而且外面增加的那一圈里面的空间一定为圈外。因此可以深搜所有圈外结点,打上标记,然后给圈内节点染色。
	
	\subsection*{P1003 蒟蒻爱滑雪}
	对于每一个节点,ans[i][j]=max(ans[di][dj]+1) (M[di][dj]<M[i][j]) M为地图,di,dj为相邻点。
	
	因为每次只需要最大的距离,所以考虑记忆化搜索。
	
	没有边界条件,因为到最小的地方四个方向都走不动。
	\subsection*{P1005 蒟蒻打扑克}
	这是一个很明显的搜索,但是52张扑克表示搜索量极大,考虑剪枝
	
	\textbf{可行性}:
	
	1.用桶把1-13的数的个数记录下来,每次搜索枚举1-13而不是所有牌。
	
	2.所有牌加起来的总和不超过364,可以先将小于等于364的13以内的约数预处理出来,之后每次搜索都枚举该数的约数,可以减少判断次数。
	
	\textbf{最优性}:
	
	1.1这个数字是万金油,是所有数字的约数,1的数量多的话会增加很多累赘运算,尤其是有解的时候,1会严重拉低效率,于是考虑从大约数到小约数倒序枚举。(有解情况的优化)
	
	2.(\textbf{最重要的剪枝})倒序搜索,从sum到0搜索,这样可以极大化减少搜索量,原理为大数的13以内的约数更少,能做到更有效的剪枝。
	
	加上这些剪枝基本上可以ac(coarse answer)
\end{document}}