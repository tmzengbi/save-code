\documentclass[UTF8]{ctexart}
\usepackage{graphicx}
\title{第三周周练题解}
\author{ZeNgBi}
\date{}
\begin{document}
	\maketitle
	\subsection*{P1000 数列编辑器}
	两个栈,一个栈存光标前的数,一个栈存光标后的数,可以用两个数组,一个动态维护光标前的最大前缀和,一个维护光标前的前缀和,查询时可以$O(1)$输出,时间复杂度$O(n)$
	
	\subsection*{P1001 堆排序}
	堆排模板题
	
	\subsection*{P1002 合并果子}
	先把所有值放入堆中,用小根堆动态维护最大值,每次取出最小两个数,相加后放入堆中维护,堆空时算法结束
	
	把每次的代价相加就是答案,时间复杂度$O(nlogn)$复杂度上界为构造小根堆
	
	\subsection*{P1003 新壳栈}
	操作1和操作2都是正常栈的操作,操作3可以暴力处理,注意异常输出。
	
	\subsection*{P1004 Zayin买礼物}
	可以先以礼物好看值为关键字从大到小排列礼物,从大到小遍历$b_i$,并维护一个小根堆,如果当前$b_i\leq d_j$那么将这个礼物放到小根堆中维护,然后取出最小值,如果堆为空取不出最小值那么无法满足。答案为每次最小值的和。
	
	\subsection*{P1005 亲戚}
	并查集板子题
	
	\subsection*{P1006 青蛙求偶记}
	求出两两点的距离求出后排序,从小到大遍历这些边,将每次边的两个顶点放入并查集中维护,当起点1和终点2在同一个并查集中时,输出边长。
	
	\subsection*{P1007 食物链}
	需要开3倍并查集分别保存A B C三类动物(1-N表示A (N+1)-(N*2)表示B (N*2+1)-(N*3)表示C)
	
	以下merge都表示合并并查集的操作,pre[ ]表示并查集数组,Find表示找爹函数。
	
	如果x和y是同类那么x,y在不冲突的前提下可能都是A,可能都是B,也可能都是C。
	merge(x,y),merge(x+N,y+N),merge(x+2*N,y+2*N) 表示三种情况下A,B都是同类
	
	如果x吃y那么x,y在不冲突的前提下可能x是A,y是B。或者x是B,y是C。或者x是C,y是A
	于是merge(x,y+N);merge(x+N,y+2*N);merge(x+2*N,y);
	
	x和y是同类时,x与y不冲突表示x不吃y而且y不吃x 就是Find(x)!=Find(y+N)且Find(y+N)!=Find(x)这里只需要判断A和B就可以,因为B对C和C对A是对称的
	
	x吃y时,x与y不冲突表示x不与y是同类而且y不吃x 那就是Find(x)!=Find(y)且Find(x+N)!=Find(y)
	
	注意x或y大于N时候的特判
\end{document}